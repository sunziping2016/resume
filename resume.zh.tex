% !TEX program = xelatex

\documentclass{resume}

\usepackage[fontset=windows]{ctex}
\usepackage[left=0.6in,top=0.4in,right=0.6in,bottom=0.4in]{geometry}
\usepackage[hidelinks]{hyperref}
\newcommand{\tab}[1]{\hspace{.2667\textwidth}\rlap{#1}}
\newcommand{\itab}[1]{\hspace{0em}\rlap{#1}}
\name{孙子平}
\address{(+86)18516279008 \\ \underline{\href{mailto:me@szp.io}{me@szp.io}} \\ \underline{\url{https://szp.io}} \\ \underline{\url{https://github.com/sunziping2016}}}

\begin{document}

\begin{rSection}{教育和奖学金}
\hspace*{-0.2in}\textbf{清华大学}~\textit{软件工程专业本科生} \hfill \emph{2015.9 - 2020.7} \\
GPA 3.67,排名20/83。获得过\textit{清华之友-光华奖学金}和\textit{清华之友-金立ELIFE三等奖学金}\\
\hspace*{-0.2in}\textbf{清华大学}~\textit{软件工程专业工学硕士} \hfill \emph{2020.9 - 2023.7} \\
导师为周旻副研究员,从事\textit{静态分析方向},主要研究值流图上的缺陷路径可达性分析
\end{rSection}

\begin{rSection}{技能}
\begin{tabular}{ @{} >{\bfseries}l @{\hspace{6ex}} l }
编程语言 & 很热爱Rust,非常熟悉C,能熟练运用C++、Python、TypeScript、Haskell等等 \\
编译原理 & 掌握并实践了LLVM IR上的控制流分析、数据流分析、值流分析的相关算法 \\
机器学习 & 能熟练运用NumPy、Pandas和PyTorch等科学计算库、以及Scrapy等爬虫框架 \\
Web前后端 & 掌握Vue、React,能使用Koa、Tornado、actix搭建后端,会使用一些SQL和NoSQL \\
其他 & 日常使用ArchLinux,混迹Geek社团,爱好打磨代码,有一定的代码洁癖
\end{tabular}
\end{rSection}

\begin{rSection}{社工和实习经历}
\hspace*{-0.2in}\textbf{软件学院学生科协}~\textit{技术部部长} \hfill \emph{2016.11 - 2018.1} \\
开发了投票、微信弹幕、抽奖等系统,并培训部员 \\
\hspace*{-0.2in}\textbf{Pony.ai}~\textit{负责自动驾驶路径规划与控制} \hfill \emph{2019.3 - 2019.8} \\
改进了issue bot,重构了测车软件 \\
\hspace*{-0.2in}\textbf{微软STCA}~\textit{深度学习用于代码检索} \hfill \emph{2019.7 - 2019.7} \\
复现了CODEnn的模型,重构了数据处理的代码 \\
\hspace*{-0.2in}\textbf{泛联新安}~\textit{值流路径可达性分析} \hfill \emph{2021.7 - 2021.7} \\
在指针分析的基础上,构建带数据依赖条件的值流图,并最终借助SMT求解器等方法确认缺陷路径可达性
\end{rSection}

\begin{rSection}{项目}
\hspace*{-0.2in}\textbf{机器学习}\\
\textbf{THUCourseSpider}:全自动刷课脚本;爬取并标注2k个验证码后,用RNN使验证码识别准确率达到95\%\\
\textbf{其他}:用C++、多线程和SIMD实现MLP;用NumPy实现CNN;本科曾参与龙明盛老师迁移学习库的编写\\
\hspace*{-0.2in}\textbf{Web前后端}\\
\textbf{YAWeChatTicket}:借助微信发布活动、抢票、检票的系统,RBAC,前后端分离,RESTful,PWA,代码破万\\
\textbf{其他}:搭配用户信息爬虫的微信墙、抽奖和弹幕系统;实时投票系统;设计并实现简易联盟链\\
\hspace*{-0.2in}\textbf{游戏}\\
\textbf{Qt-PlantsVsZombies}:用Qt编写的高仿真植物大战僵尸游戏\\
\textbf{其他}:用于科展的体感游戏;横版物理跑酷游戏;联网井字棋和联网贪吃蛇;躲避类HTML5小游戏;扫雷外挂\\
\hspace*{-0.2in}\textbf{算法及底层系统}\\
\textbf{zhihu-search-engine}:手写动态数组、链表、哈希表及容错快速的HTML解析器,使用倒排索引检索网页\\
\textbf{ftp-server}:非阻塞IO,可通过CLI管理用户,基于bcrypt的密码检查,双栈,干净的信号处理,防御式编程\\
\textbf{rsa-rs}:借助汇编手写高精度整数,实现了RSA密钥的生成、加密和解密,手写了openssh密钥格式的解析\\
\textbf{ray-tracing}:多线程和AoSoA SIMD的光线追踪,借助泛型切换SIMD指令集,导出Python FFI\\
\textbf{SoundMessage/Localization}:OFDM+QPSK调制声波传输信息,采用FMCW定位,提供Android APP\\
\textbf{Tsmart}:导师带领的静态分析工具,在读了代码后我成为主要维护人员,修复了很多bug并做了些改进\\
\textbf{其他}:汇编还原画图;小型数据库;函数式解释器;带领40人改进某OS;我的静态分析工具Tilly
(开发中)
\end{rSection}

\end{document}
