% !TEX program = xelatex

%%%%%%%%%%%%%%%%%%%%%%%%%%%%%%%%%%%%%%%%%
% Medium Length Professional CV
% LaTeX Template
% Version 2.0 (8/5/13)
%
% This template has been downloaded from:
% http://www.LaTeXTemplates.com
%
% Original author:
% Rishi Shah
%
% Important note:
% This template requires the resume.cls file to be in the same directory as the
% .tex file. The resume.cls file provides the resume style used for structuring the
% document.
%
%%%%%%%%%%%%%%%%%%%%%%%%%%%%%%%%%%%%%%%%%

\documentclass{resume}

\usepackage{ctex}
\usepackage[left=0.6in,top=0.4in,right=0.6in,bottom=0.4in]{geometry}
\usepackage[hidelinks]{hyperref}
\newcommand{\tab}[1]{\hspace{.2667\textwidth}\rlap{#1}}
\newcommand{\itab}[1]{\hspace{0em}\rlap{#1}}
\name{孙子平}
\address{(+86)18516279008 \\ \underline{\href{mailto:me@szp.io}{me@szp.io}} \\ \underline{\url{https://szp.io}} \\ \underline{\url{https://github.com/sunziping2016}}}

\begin{document}

\begin{rSection}{教育和奖学金}
{\bf 清华大学}~\textit{软件学院本科生} \hfill {\em 2015.9 - 2020.7}\\
GPA 3.67,排名20/83。获得过\textit{清华之友-光华奖学金}和\textit{清华之友-金立ELIFE三等奖学金}。
\end{rSection}

\begin{rSection}{技能}
\begin{tabular}{ @{} >{\bfseries}l @{\hspace{6ex}} l }
编程语言 & 精通C,能熟练运用C++、Python、JavaScript(Node.js)、Haskell,使用过汇编 \\
机器学习 & 能熟练运用NumPy、Pandas、MatplotLib、sk-learn和PyTorch,使用过TensorFlow \\
前后端 & 能熟练运用Vue、Koa、Express,使用过React、Three.js和Django \\
爬虫 & 能熟练运用Scrapy和BeautifulSoap \\
C++ & 能熟练运用Boost、Qt \\
Linux & 日常使用ArchLinux,能熟练配置LNMP、MongoDB、Docker等等
\end{tabular}
\end{rSection}

\begin{rSection}{社工和实习经历}
{\bf 软件学院学生科协}~\textit{技术部部长} \hfill {\em 2016.11 - 2018.1} \\
开发了投票、微信弹幕、抽奖等系统,并培训部员 \\
{\bf Pony.ai}~\textit{负责自动驾驶路径规划与控制} \hfill {\em 2019.3 - 2019.8} \\
改进了issue bot,重构了测车软件。\\
{\bf 微软STCA}~\textit{深度学习用于代码检索} \hfill {\em 2019.7 - 2019.7} \\
复现一个叫CODEnn的模型,重构了数据处理的代码。
\end{rSection}

\begin{rSection}{项目}
\textit{以下项目只是简介。详细内容请移步\underline{\url{https://szp.io/2018/12/04/my-projects/}}或GitHub。}\\
\hspace*{-0.2in}\textbf{机器学习}\\
\textbf{VeryTinyCnn}:只有前向传播的卷积神经网络实现,使用了标准C++11和多线程及SIMD指令优化。\\
\textbf{Minimal-NeuralNet}:多层反向传播的神经网络,只使用了标准C++。\\
\textbf{transfer-tensorflow}:使用TensorFlow编写的迁移学习库。\\
\hspace*{-0.2in}\textbf{前后端}\\
\textbf{YAWeChatTicket}:借助微信发布活动、抢票、检票的系统,有完整的用户权限系统,前后端分离,RESTful API,前端为PWA,总代码量破万。\\
\textbf{wewall}:包含微信墙和桌面弹幕,支持emoji,包含一个爬虫自动模拟微信管理员登录爬取用户头像,被学生节使用。\\
\hspace*{-0.2in}\textbf{游戏}\\
\textbf{Qt-PlantsVsZombies}:用Qt编写的植物大战僵尸游戏,代码量4.5k。\\
\textbf{Collision}基于Leap Motion的体感游戏,架构采用了MVC,用Java编写。\\
\hspace*{-0.2in}\textbf{算法及底层系统}\\
\textbf{zhihu-search-engine}:重新实现了C++中vector、list、unordered\_set等数据结构,手写了一个极快的、有良好异常处理的HTML parser,界面使用了Electron。\\
\textbf{GraphVisualization}:部分调用了Boost图的算法,前端使用了Three.js 实现可视化。\\
\textbf{WinminePlugin}直接读取扫雷内存,再发送窗口消息到扫雷的外挂,高一写的,这是我最早写的程序。\\
\textbf{xv6-improved}:我带领40人的团队,一起改进一个教学用的操作系统。
\end{rSection}

\end{document}
