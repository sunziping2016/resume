% !TEX program = xelatex

\documentclass{resume}

\usepackage[fontset=windows]{ctex}
\usepackage[left=0.6in,top=0.4in,right=0.6in,bottom=0.4in]{geometry}
\usepackage[hidelinks]{hyperref}
\newcommand{\tab}[1]{\hspace{.2667\textwidth}\rlap{#1}}
\newcommand{\itab}[1]{\hspace{0em}\rlap{#1}}
\name{孙子平}
\address{(+86)18516279008 \\ \underline{\href{mailto:me@szp.io}{me@szp.io}} \\ \underline{\url{https://szp.io}} \\ \underline{\url{https://github.com/sunziping2016}}}

\begin{document}

\begin{rSection}{教育和奖学金}
\hspace*{-0.2in}\textbf{清华大学}~\textit{软件工程专业本科生} \hfill \emph{2015.9 - 2020.7} \\
GPA 3.67,排名20/83。获得过\textit{清华之友-光华奖学金}和\textit{清华之友-金立ELIFE三等奖学金} \\
\hspace*{-0.2in}\textbf{清华大学}~\textit{软件工程专业工学硕士} \hfill \emph{2020.9 - 2023.7} \\
导师为周旻副研究员,从事\textit{静态分析方向},主要研究值流图上的缺陷路径可达性分析
\end{rSection}

\begin{rSection}{技能}
\begin{tabular}{ @{} >{\bfseries}l @{\hspace{3ex}} l }
编程语言 & 热爱Rust,能熟练运用C/C++、Java、Python、TypeScript、Haskell、Verilog等等 \\
机器学习 & 能熟练运用NumPy、Pandas和PyTorch等科学计算库、以及Scrapy等爬虫框架 \\
Web前后端 & 掌握Vue、React,能使用Koa、Tornado、tokio等搭建后端,会使用一些SQL和NoSQL \\
团队协作 & 几乎所有团队项目的组长,有较强的协调能力,熟练部署CI并运用看板,善于维护文档 \\
其他 & ArchLinux用户,混迹Geek社团,爱好打磨代码,外向友善,理想是编写优秀代码分析工具
\end{tabular}
\end{rSection}

\begin{rSection}{社工和实习经历}
\hspace*{-0.2in}\textbf{软件学院学生科协}~\textit{技术部部长} \hfill \emph{2016.11 - 2018.1} \\
开发了投票、微信弹幕、抽奖等系统,并培训部员 \\
\hspace*{-0.2in}\textbf{微软STCA}~\textit{深度学习用于代码检索} \hfill \emph{2019.7 - 2019.7} \\
复现了CODEnn模型,重构了数据处理的代码 \\
\hspace*{-0.2in}\textbf{其他}:\textbf{Pony.ai}~\textit{改进测车软件}~~\textbf{商汤科技}~\textit{入职即离职}~~\textbf{旷视}~\textit{参与张量优化算法}~~\textbf{泛联新安}~\textit{编写代码分析}
\end{rSection}

\begin{rSection}{项目}
\hspace*{-0.2in}\textbf{机器学习} \\
\textbf{THUCourseSpider}:全自动刷课脚本;爬取并手动标注数据集后,用RNN使验证码识别准确率达到95\% \\
\textbf{其他}:二次元头像生成;用C++、多线程和SIMD实现MLP;参与龙明盛老师迁移学习库的编写 \\
\hspace*{-0.2in}\textbf{Web前后端} \\
\textbf{YAWeChatTicket}:借助微信发布活动、抢票、检票的系统,RBAC,RESTful,PWA,代码破万 \\
\textbf{其他}:搭配用户信息爬虫的微信墙、抽奖和弹幕系统;实时投票系统;设计并实现简易联盟链 \\
\hspace*{-0.2in}\textbf{游戏} \\
\textbf{Qt-PlantsVsZombies}:用Qt编写的高仿真植物大战僵尸游戏,B站\texttt{BV1ub4y1v7MF} \\
\textbf{其他}:体感游戏;横版物理跑酷游戏;联网井字棋和联网贪吃蛇;躲避类HTML5小游戏;扫雷外挂 \\
\hspace*{-0.2in}\textbf{算法及底层系统} \\
\textbf{zhihu-search-engine}:手写动态数组、链表、哈希表及容错快速的HTML解析器,使用倒排索引检索网页 \\
\textbf{ftp-server}:基于epoll的非阻塞IO,可通过CLI管理用户,基于bcrypt的密码检查,双栈,信号处理 \\
\textbf{ray-tracing}:多线程和AoSoA SIMD的光线追踪,借助泛型切换SIMD指令集,导出Python FFI \\
\textbf{SoundMessage/Localization}:OFDM+QPSK调制声波传输信息,采用FMCW定位,提供Android APP \\
\textbf{其他}:实现RSA;汇编画图;小型数据库;函数式解释器;我的静态分析工具(开发中)
\end{rSection}

\begin{rSection}{科研经历}
Cao, Z., \textbf{Sun, Z.}, et al. (2018, October) Deep Priority Hashing. In \textit{the 26th ACM MM} \\
\textbf{Sun, Z.}, Zhou, M. (Submitted for publication) Fast and Incremental Algorithm for Dominator Tree
\end{rSection}

\end{document}
